\documentclass[11pt]{article}
\usepackage{fullpage}
\usepackage{microtype}
\usepackage{amsmath}
\usepackage{amssymb}
\usepackage[usenames,dvipsnames]{color}
\usepackage{float} % lets you have non-floating floats
\usepackage{graphicx}
\usepackage{url} % for typesetting urls
\usepackage{listings}
\usepackage{grace}
\usepackage{varioref}
\usepackage{balance}
\usepackage{hyperref}
\usepackage{url}
\urlstyle{sf}
%\textheight 8.5in  %remove it !!!

\usepackage{alltt}
%
%
%    LaTeX  Macro File  /usr2/jcm/tex/macros.tex
%
%
\tracingonline=0 % shorter error messages (on screen)

\long\def\comment#1{} % mulit-line comments
\newcommand{\note}[1]{\fbox{#1}}

% font changes  (as function calls, scribe style)

\renewcommand{\i}[1]{{\it #1\/}}       % italics with space correction
%\newcommand{\emex}[1]{\/{\em #1}}   % emphasis in example, exercise, theorem, etc.
\renewcommand{\c}[1]{{\sc #1}}       % small caps (eliminates \c as cedilla)
%\newcommand{\r}[1]{{\rm #1}}            % for roman font in math mode
%\newcommand{\s}[1]{{\scr #1}}          % script (for use with tatex) 
\newcommand{\s}[1]{{\cal #1}}
\renewcommand{\b}[1]{{\bf #1}}           % bold face
\newcommand{\calg}[1]{{\cal #1}}          % caligraphic 

\newcommand{\q}[1]{``#1''}    % matching quotes for in-line quotation

% numbered environments

%\newcounter{partcounter}
%\setcounter{partcounter}{0}
%\renewcommand{\part}[1]{\newpage \addtocounter{partcounter}{1}
%\noindent{\Large \bf Part \Roman{partcounter}. #1 } \\[1ex]}

\newtheorem{thm}{Theorem}[section]
\newtheorem{theorem}[thm]{Theorem}
\newtheorem{lemma}[thm]{Lemma}
\newtheorem{cor}[thm]{Corollary}
\newtheorem{corollary}[thm]{Corollary}
\newtheorem{claim}[thm]{Claim}
\newtheorem{prop}[thm]{Proposition}
\newtheorem{conj}[thm]{Conjecture}
\newtheorem{definition}[thm]{Definition}
\newtheorem{exercize}[thm]{Exercise}   %% fix later??
\newtheorem{example}[thm]{Example}
\newtheorem{remark}[thm]{Remark}
\newtheorem{open}[thm]{Open Problem}

%\newcommand{\proof}{\\{\bf Proof.}\ }
\newenvironment{proof}{{\bf Proof. }}{\thmbox}

%%% tabmath environment: like tabbing, but everything is in math mode
\newenvironment{tabmath}{\begingroup
\def\@startfield{\global\setbox\@curfield\hbox\bgroup\begin{math}}
\def\@stopfield{\end{math}\egroup}
\begin{tabbing}}{\end{tabbing}\endgroup}

%%% displaytabmath:  like tabmath, but centered using displaymath:

\newenvironment{displaytabmath}{%
        \begin{displaymath}\vbox\bgroup\begin{tabmath}}{%
        \end{tabmath}\egroup\end{displaytabmath}}

% axiom and inference rule (centered, in math mode,  with name at left)
\newcommand{\axiom}[2]
{\[ \hbox to \columnwidth
    { \rlap{$#2$} \hfil {$ #1 $} \hfil }
\]}

% axiom and inference rule (centered, in math mode,  with name at left)
\newcommand{\axiombig}[2]
{\[ \hbox to \columnwidth
    { $#2$ \hfil {$ #1 $} \hfil }
\]}

% axiom and inference rule (centered, in math mode,  with name at left)
\newcommand{\axiomtw}[2]
{\[ \hbox to \textwidth
    { \rlap{$#2$} \hfil {$ #1 $} \hfil }
\]}

\newcommand{\infrule}[4]
{\[ \hbox to \columnwidth %\textwidth
    { \rlap{$#4$} \hfil $
      \frac {\strut\displaystyle #1 } {\strut\displaystyle #2 } \; \rlap{$#3$}$
      \hfil }
\]}
\newcommand{\infruletw}[4]
{\[ \hbox to \textwidth
    { \rlap{$#4$} \hfil $
      \frac {\strut\displaystyle #1 } {\strut\displaystyle #2 } \; \rlap{$#3$}$
      \hfil }
\]}

% Slightly different format to prevent overlap of rule label with the
% actual rule, in the case of ``big'' rules like Update, Extend

\newcommand{\infrulebig}[4]
{\[ \hbox to \columnwidth %\textwidth
    { $#4$ \hfil $
      \frac {\strut\displaystyle #1 } {\strut\displaystyle #2 } \; \rlap{$#3$}$
      \hfil }
\]}

% Slightly different format to prevent overlap of rule label with the
% actual rule, in the case of ``big'' rules like Update, Extend

\newcommand{\infrulebigtw}[4]
{\[ \hbox to \textwidth
    { $#4$ \hfil $
      \frac {\strut\displaystyle #1 } {\strut\displaystyle #2 } \; \rlap{$#3$}$
      \hfil }
\]}

%\newcommand{\infrule}[4]
%{\[ \hbox to \columnwidth
%    { \rlap{$#4$} \hfil $
%      {{\displaystyle\strut #1}\over{\displaystyle\strut #2}}\quad\makebox[0pt][l]{\it #3} $
%      \hfil }
%\]}

% axiom and inference rule macros for use in tables, etc.
% presumed in math mode #1=axiom, #2=side condition
\newcommand{\Axiom}[2]{
{\displaystyle\strut #1 }\qquad\makebox[0pt][l]{\it #2}
}
% presumed in math mode #1=top, #2=bottom, #3=side condition
\newcommand{\Infrule}[3]{
{{\displaystyle\strut #1}\over{\displaystyle\strut #2}}\qquad\makebox[0pt][l]{\it #3}
}

% sequence of #1's, numbered up to #2  (e.g., seq{x}{n} for x1, ..., xn   )

\newcommand{\seq}[2]{#1_{1} \ldots #1_{#2}}  
\newcommand{\ie}{{\it i.e.}}
\newcommand{\eg}{{\it e.g.}}
\newcommand{\cf}{{\it c.f.\,}}

% common symbols

\newcommand{\union}{\cup}
\newcommand{\intersect}{\cap}
\newcommand{\subs}{\subseteq}
\newcommand{\el}{\in}
\newcommand{\nel}{\not\in}
\newcommand{\ns}{\emptyset}
\newcommand{\compose}{\circ}
\newcommand{\set}[2]{ \{\, #1 \,\mid\, #2 \,\}  } % set macro
\newcommand{\infinity}{\infty}
\newcommand{\pair}[1]{\langle #1 \rangle}
\newcommand{\tuple}[1]{\langle #1 \rangle}
\newcommand{\da}{\downarrow}


\newcommand{\fa}{\forall}
\newcommand{\te}{\exists}
\newcommand{\imp}{\supset}
%\newcommand{\implies}{\supset}  conflicts with other def in other 
%file
\newcommand{\ts}{\vdash}
\newcommand{\dts}{\models}

\newcommand{\aro}{\mathord\rightarrow} % see pages 154-155 of TeX manual
\newcommand{\paro}{\rightharpoonup} 
\newcommand{\karo}{\mathop\Rightarrow} % see pages 154-155 of TeX manual
\newcommand{\cross}{\times}
\newcommand{\dlb}{\lbrack\!\lbrack}
\newcommand{\drb}{\rbrack\!\rbrack}
\newcommand{\mean}[1]{\lbrack\!\lbrack #1 \rbrack\!\rbrack}
\newcommand{\mnr}[1]{\lbrack\!\lbrack #1 \rbrack\!\rbrack \rho}
\newcommand{\lam}{\lambda}
\newcommand{\subst}[2]{{}[#1/#2]}
\renewcommand{\dot}{\mathrel{\bullet}}
\newcommand{\Dinf}{D_{\infty}}
\newcommand{\bottom}{\perp}

% commands for semantics in inheritance paper
\newcommand{\mmn}[1]{\lbrack\!\lbrack #1 \rbrack\!\rbrack^M \rho}
\newcommand{\cmn}[1]{\lbrack\!\lbrack #1 \rbrack\!\rbrack^C \rho}
\newcommand{\mc}{\Xi^M}
\newcommand{\cc}{\Xi^C}

\mathcode`:="603A  % treat : as punctuation instead of relation in math mode
\mathchardef\colon="303A        % relation colon

\newcommand{\eqdef}{\mathrel{:=}}
\newcommand{\Dom}{\mathop{\rm dom}}
\newcommand{\Pow}{\mathop{\rm Pow}}

\newcommand{\aequiv}{\equiv_\alpha}
\newcommand{\baro}{\buildrel \beta \over \rightarrow}
\newcommand{\earo}{\buildrel \eta \over \rightarrow}
\newcommand{\red}{\rightarrow\!\!\!\!\rightarrow}
\newcommand{\backred}{\leftarrow\!\!\!\!\leftarrow}
\newcommand{\ared}{\buildrel \alpha \over \red}
\newcommand{\bred}{\buildrel \beta \over \red}
\newcommand{\dred}{\buildrel \delta \over \red}
\newcommand{\ered}{\buildrel \eta \over \red}
\newcommand{\conv}{\leftrightarrow}
\newcommand{\beconv}{\buildrel {\beta, \eta} \over\leftrightarrow}

% lambda calculus abbreviations
\newcommand{\letdec}[3]{\b{let\ } #1 = #2 \b{\ in\ } #3}
\newcommand{\letrec}[3]{\b{letrec\ } #1 = #2 \b{\ in\ } #3}

\newcommand{\thmbox}
   {{\ \hfill\hbox{%
      \vrule width1.0ex height1.0ex
   }\parfillskip 0pt }}
\newcommand{\qed}{\thmbox}


% make single spacing

\newcommand{\singlespace}{\renewcommand{\baselinestretch}{1}\@normalsize}
\newcommand{\etal}{{\em et al.}}

% bycase command for definition by cases (pg 49 of LaTeX)

\newcommand{\bycase}[1]
        {\left\{ \begin{array}{ll}  #1  \end{array} \right. }


% make \cite put blanks after the comma (use in alpha style)

\def\@citex[#1]#2{\if@filesw\immediate\write\@auxout{\string\citation{#2}}\fi
  \def\@citea{}\@cite{\@for\@citeb:=#2\do
    {\@citea\def\@citea{,\penalty100\hskip2.5pt plus1.5pt minus.8pt}%
       \@ifundefined{b@\@citeb}{{\bf ?}\@warning
       {Citation `\@citeb' on page \thepage \space undefined}}
\hbox{\csname b@\@citeb\endcsname}}}{#1}}

\newcommand{\change}[2]{#2}
\newcommand{\rjust}{\hspace*{\fill}}
%\input{rmacros}
\newcommand{\tst}{\vdash}
\newcommand{\bs}{\backslash}
\newcommand{\self}{\rm{\texttt{self}}}
\newcommand{\si}{\rm{\texttt{selfinst}}}
\newcommand{\sm}{\rm{\texttt{self}}}
\newcommand{\selfm}{{\tt{self}}} %new
\newcommand{\selfi}{{\tt{inst}}} %new
\newcommand{\err}{{\rm\texttt{error}}}
\newcommand{\ra}{\rightarrow}
\newcommand{\leI}{{\rm \, <\!\!\!\# \, }}
\newcommand{\leS}{{\rm \ <: \ }}
\newcommand{\iscom}{{\rm{COMMAND}}}
\newcommand{\isprog}{{\rm PROGRAM}}
\newcommand{\com}{{\texttt{command}}}
\newcommand{\Com}{{\texttt{COMMAND}}}
\newcommand{\Prog}{{\texttt{PROGRAM}}}
\newcommand{\semconf}{{\rm \ {\it conf}\ }}
\newcommand{\conf}{{\rm \ {\bf conf}\ }}
\newcommand{\sext}{{\rm\ {\it ext}\ }}
\newcommand{\error}{\texttt{error}}
\newcommand{\ST}{\texttt{InstType}}
\newcommand{\me}{\texttt{self}}
\newcommand{\supermeth}{\rm{\texttt{super}}}
\newcommand{\super}{{\texttt{super}}}
\newcommand{\rf}{{\texttt{ref}}}
\newcommand{\supervar}{{\texttt{supervar}}}
\newcommand{\MT}{{\texttt{MyType}}}
\newcommand{\fn}{{\texttt{fun}}}
\newcommand{\Tobj}{{\texttt{ObjectType}}}
\newcommand{\TobjV}{{\texttt{VisObjType}}}
\newcommand{\Tclass}{{\texttt{ClassType}}}
\newcommand{\ObjectType}{{\texttt{ObjectType}}}
\newcommand{\ClassType}{{\texttt{ClassType}}}
\newcommand{\TP}{{\texttt{TYPE}}}
\newcommand{\RTM}{{\rm{\it RecToMem}}} %new
\newcommand{\RM}{{\rm{\it RecToMem}}}
\newcommand{\class}{{\texttt{class}}}
\newcommand{\extend}{{\texttt{extend}}}
\newcommand{\by}{{\rm{\it by}}}
\newcommand{\update}{{\texttt{update}}}
\newcommand{\with}{{\texttt{with}}}
\newcommand{\new}{{\texttt{new}}}
\newcommand{\jf}{{\texttt{if}}}
\newcommand{\thn}{{\texttt{then}}}
\newcommand{\els } {{\texttt{else}}}
\newcommand{\fun}{{\texttt{fun}}}
\newcommand{\gts}{{\texttt{ gets}}}
\newcommand{\cls}{{\texttt{close}}}
\newcommand{\cet}{C,E \ts}
\newcommand{\ovrd}{\oplus}
\newcommand{\tyerr}{{\texttt{tyerr}}} %new
\newcommand{\nc}{{\textit{\scriptsize{Comp}}}} %new

\newcommand{\ed}[1]{{#1}_{\rho}}
\newcommand{\edz}[1]{{#1}_{\rho,0}}
\newcommand{\edp}[1]{{#1}_{\rho'}}
\newcommand{\edpz}[1]{{#1}_{\rho',0}}
\newcommand{\edf}[1]{{#1}_{\rho_f}}
\newcommand{\fntp}[2]{func(#1): #2}
\newcommand{\tsm}{\vdash_{M}}
\newcommand{\tsa}{{\ts_{\! \cal A}\ }}


\author{Kim B. Bruce}
\date{\today}
\title{Semantics of Grace classes}


\begin{document}
\bibliographystyle{alpha}
\maketitle

Here we sketch the definition of the semantics of classes and objects
in Grace.  The semantics is presented in terms of a (typed) lambda
calculus with records and tuples.  Consider the following simple 
class definition:

\begin{lstlisting}
class C.new(a:U)->V{
   var x:W := b
   method m(s:S)->T {
      e
   }
}
\end{lstlisting}

We have only included a single instance variable and a single method
for simplicity.  (We could generalize this by writing overscores
over all the components.)  We have also left off definitions and
start-up code, again to keep the analysis simple (though both
are pretty simple to add).

Also, for simplicity, we assume that the expression b contains no 
references to self (implicit or explicit).

We will use the notation $\mnr{e}$ for the meaning of expression $e$ 
with respect to environment $\rho$ (which keeps track of the meaning 
of all identifiers in the current scope).

We can obtain the meaning of class $C$ as follows:
$$\begin{array}{l@{}l}
\mnr{C} = \{new = \lambda a' \in U.<&\{xInit = 
\mnr{\texttt{b}}[a'/\texttt{a}]\}, \\
&\lambda instLocn.\lambda \textit{self}
\in V.\{m = \lambda
s':S.\mnr{e}[a'/\texttt{a},instLocn/\texttt{x},\textit{self}/\texttt{self},s'/\texttt{s}]\}>\}
\end{array}$$

Let's unpack the definition.  First, in the semantics I am using curly
brackets for records (i.e., where fields have names) and 
$<>$ for tuples.

The meaning of a class is a record with a single field with name 
corresponding to the name of the constructor, in this case, 
\texttt{new}.  That field corresponds to a function with the same 
parameters as the constructor (I've tagged all semantic identifiers 
with a `` ` `` to distinguish them from the corresponding identifiers 
in the language syntax).

That top-level function, when provided with the appropriate parameters, returns
a pair consisting of a record of initial values for instance variables
and a function which will represent the meanings of
methods.  The first record consists of the initial values for instance
variables.  In the case above, I've provided the initial value for
\texttt{x} with label \textit{xInit}.  When the constructor is applied
we will need to allocate a location for that variable, but for now we
only need to keep track of the initial value.

We can't completely determine the meaning of the methods without the
location of the instance variables and the meaning of \texttt{self}.
Thus the meaning of methods takes these as parameters and then returns
a record with the meanings of the methods.  The methods are of course
functions of their parameters.  Notice that the bodies of the methods
are evaluated in environments updated to provide the values of the
parameters to the constructor, locations of the instance variables,
the meaning of \texttt{self}, and the meanings of the method parameters.

What about the object generated by the constructor?  That's not too 
bad.  I'll write it in terms of the code above rather than obtaining 
the pieces indirectly from the meaning of \texttt{C} (though that 
would be the way to write the formal definition).


$$\mnr{C.new(a)} = <\{x = l\},fix(\lambda \textit{self}
\in V.\{m = \lambda
s':S.\mnr{e}[a'/\texttt{a},l/\texttt{x},\textit{self}/
\texttt{self},s'/\texttt{s}]\}>\}$$

\noindent where $a' = \mnr{a}$ and \textit{l} is a new location whose 
initial value is given by $\mnr{\texttt{b}}[a'/\texttt{a}]$.

Thus the meaning of the object generated by \texttt{C.new} is a pair, 
where the first item is a record containing the newly-allocated 
locations of the instance variables, each of which is initialized 
with the initial value specified in the class definition.

The second item is the record of methods.  These are first updated
with the locations of the newly generated instance variables and then
the fixed point is taken to make sure that \texttt{self} refers to the
entire method suite.

I'm going to stop here rather than going on to explain inheritance as 
I have to meet my daughter in downtown LA this evening.  However, 
hopefully staring at this for a while will help in understanding when 
the various items are evaluated (notice almost all valuations are 
hidden under lambdas, and hence evaluated only after they are applied 
to the appropriate values -- recall we never do evaluations under 
lambdas if we want side-effects to happen in the right order.


\end{document}